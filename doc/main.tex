\documentclass[11pt,a4paper]{article}
\usepackage[]{inputenc}
\usepackage[T1]{fontenc}
\usepackage{fullpage}
\usepackage[color]{coqdoc}
\usepackage{amsmath,amssymb}

%\usepackage[backend=biber]{biblatex}
\usepackage[backend=bibtex]{biblatex}

\usepackage{draftwatermark}
%\usepackage[final]{draftwatermark}
\SetWatermarkScale{5}
\SetWatermarkLightness{0.95}

\usepackage{hyperref}
\usepackage{amsthm}
\usepackage{verbatim}
\usepackage{alltt}
\usepackage{fullpage}
\usepackage{graphicx}
\usepackage{wrapfig}
%\usepackage{stmaryrd} %for bigsqcap
%\usepackage{mathpartir} % http://pauillac.inria.fr/~remy/latex/mathpartir.sty

\bibliography{references}{}
%\bibliographystyle{plain}

\newcommand \Hstar {{\ensuremath{\mathcal H^\ast}}}

\newcommand \TODO [2] [TODO] {{\fbox{\textbf{\small #1} #2}}}
%\newcommand \TODO [2] [] {}

\begin{document}

\title{ The Inadvertantly Typed Pure $\lambda$-join-Calculus }
\author{ Fritz Obermeyer }

\maketitle

\begin{abstract}
Typed $\lambda$-calculi enjoy a number of convenient properties not
satisfied by untyped $\lambda$-calculi.
We show that one family of untyped $\lambda$-calculi 
Specifically we show that in pure untyped $\lambda$-calculi
extended with a join operation
and obeying Hyland and Wadsworth's \Hstar axiom,
a rich but inconsistent type system is definable within the calculus
using Dana Scott's types-as-closures idiom.
We develop proofs in Coq of a number of well-typedness properties
of two systems:
the pure $\lambda$-join calculus modeling nondeterministic computation
and the stochastic $\lambda$-join calculus modeling computaion with
convex sets of probability distributions.
\end{abstract}

\newpage

\tableofcontents

\newpage

\input{body.tex}

\printbibliography
\end{document}
